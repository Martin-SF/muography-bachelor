\chapter{Zusammenfassung und Ausblick}

% \marktodo{Forschungsfrage beantworten und einordnen: tatsächliche Aussagekraft und Einschränkungen\\
% inwiefern sind die aussagen aus dieser arbeit übertragbar auf die realword} 
% diskussion:

In dieser Arbeit wurde anhand einer Bohrung aus dem Ruhrgebiet ein Bodenmodell erstellt,
um Detektorzählraten bei verschiedenen Wasserständen innerhalb
für den Bergbau trocken gelegten Erdschichten zu simulieren.
Es konnten signifikante Unterschiede in den Zählraten gezeigt werden.
Pro \SI[]{100}[]{m} ändert sich die Detektorrate um \SI[]{1,55}[]{\%}.

% Die ausgerechneten Messzeiten um einen Wasserstand mit einer Genauigkeit von
% \SI[]{100}[]{m} mithilfe des beschriebenen Detektor auflösen zu
% können, beträgt \SI[]{01010101}[]{Tage} .....  \marktodo{spezifizieren}

Da sich in der Realität die Wasserstände um \si[]{cm} pro Woche ändern können,
detektiert der in dieser Arbeit angenommene Detektor zu wenig Teilchen pro Tag,
um Veränderungen schnell genug oder überhaupt verlässlich nachweisen zu können.

Da in dieser Arbeit nur die Detektion über die horizontale Oberfläche in Betracht gezogen wird,
könnte in zukünftigen Arbeiten über die Vergrößerung des Detektors in vertikaler Richtung
die Messzeit gesenkt werden, technisch ist lediglich zu beachten, dass der Detektor in 
\SI[]{30}[]{Fuß} Schritten teilbar sein muss. 
Des Weiteren sollte über Alternativen zur Platzierung des Detektors innerhalb
einer Bohrung nachgedacht werden. Der Durchmesser von \SI[]{10}[]{cm} 
limitiert die Möglichkeiten des Detektors massiv.

Alte Bergwergschächte hatten ursprünglich einen Durchmesser von \num[]{7} bis \SI[]{10}[]{m}.
Dort existieren teilweise Inspektionsrohre mit einem Durchmesser von \num[]{80} bis \SI[]{100}[]{cm}.
Diese werden vorgehalten, um ggf. Pumpen einzuhängen. Jene Inspektionsrohre würden alleine durch den höheren
Durchmesser eine ca. \num[]{24} mal größere Grundfläche besitzen können, welche
bspw. die Rate für \SI[]{800}[]{m} von 
\SI[]{0.254}[]{\frac{\mathrm{Myonen}}{\mathrm{Tag}}} auf 
\SI[]{6.108}[]{\frac{\mathrm{Myonen}}{\mathrm{Tag}}} heben würde.




Wenn die xy-Ebene einer Bohrung verlassen wird, treten große Inhomogenitäten auf.
Dies senkt die Übertragbarkeit des Bodenmodells auf die echte Welt.
Um diese zu verbessern, könnten mit mehr Bohrungen das Bodenmodell in der xy-Ebene verfeinert werden,
um die Berechnung zu verbessern.

Spannend könnte im Rahmen kommender Arbeiten eine echte 
Messung innerhalb eines Bohrlochs oder anderen Öffnungen sein.
Es könnte untersucht werden, inwiefern die in dieser Arbeit bestimmten
relativen Raten-Unterschiede mit den Vorhersagen dieser Arbeit übereinstimmen.

Auch könnte ein Vergleich der absoluten Detektorraten mit den simulierten
ein Maß über die Gültigkeit aller Näherungen in dieser Arbeit liefern. 

% energiethreshold zur verbesserung der aussaagekraft